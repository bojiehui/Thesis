\chapter{Introduction}
\label{1.0 Intr}

The future Internet will be the Internet of Things, meaning that everything from household applications and appliances to industrial facilities and machinery will be connected to the Internet. Over the last few years there has been a concentrated effort to connect various peripherals to the Internet, and connecting sensors was among one of them. Sensors, unlike many larger devices, present a unique challenge to network specialists due to the many constraints and limitations reside in them. Such constraints include, but are not limited to, small memory size, limited power, as well as the diversity of communication protocols used in Wireless Sensor Networks (WSNs)\@. Until the appearance of IETF IPv6 over Low Power Personal Area Network (6LoWPAN)\@, the successful attainment of this task was questionable. 6LoWPAN proposes to unify network protocols so that the complexity of protocol translation is eliminated. 

In 6LoWAPNs, the links between sensor nodes are lossy and support low data rate. Furthermore, such networks may potentially comprise of to thousands of nodes. These characteristics offer unique challenges to a routing solution. The IETF ROLL Working Group has defined the IPv6 Routing Protocol for Low power and lossy networks (RPL) to meet the requirements in \cite{draft-ietf-roll-rpl-19}.

This thesis will introduce 6LoWPAN and its characteristics, and will focus on the performance of the routing protocol in 6LoWPAN sensor networks. The reason for the thesis is in recognition that a proper simulation is a basic, yet crucial requirement of evaluating such a network. The performance thereof will be discussed based on the simulation results.

\section{Motivation}
\label{Intr:Motiv}

Several implementations of 6LoWPAN have been developed over the years, such as uIPv6 \cite{uIPv6}, BLIP \cite{BLIP}, Sensinode's NanoStack \cite{Sensinode}, Jennic's 6LoWPAN \cite{Jennic} and Nivis ISA100.11a \cite{Nivis}. The first two are open-source protocol stacks for the embedded operating systems Contiki and TinyOS. The latter are commercial protocol stacks. In this thesis, the implementation BLIP is used. BLIP version 1.0 (BLIP 1.0) has been simulated and evaluated with simple scenarios before. Recently BLIP has been updated to version 2.0 \cite{BLIP2.0}\@. In BLIP 2.0, several changes have been made in order to support the relevant IETF standards, such as header compression for 6LoWPAN in~\cite{draft-ietf-6lowpan-hc-15}, routing protocol for 6LoWPAN in~\cite{draft-ietf-roll-rpl-19}, neighbor discovery for 6LoWPAN in \cite{draft-ietf-6lowpan-nd-17}. 
Therefore the simulation had to be re-enabled for further studies of BLIP 2.0. 

In BLIP 2.0, a new routing protocol - Routing Protocol for Low power and lossy networks (RPL) replaces the Hybrid Routing Protocol for Lossy and Low Power Networks (HYDRO) as the routing protocol. The performance of RPL observed in simulation will be evaluated in this thesis.
\section{Outline}
\label{Intr:Outline}

Chapter 2 gives the general concept of WSNs. An introduction to 6LoWPAN along with its usages and implementations can be found therein. Chapter 3 presents the relevant IETF standards for 6LoWPAN. Chapter 4 discusses BLIP - the implementation of 6LoWPAN. Chapter 5 focuses on the simulation itself,  including the simulator, simulation setups and procedure. Chapter 6 gives the evaluation of the simulation results. The final chapter concludes the work, and gives the outlook for future research.
